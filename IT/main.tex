% -- Encoding UTF-8 without BOM
% -- XeLaTeX => PDF (BIBER)

\documentclass[]{cv-style}          % Add 'print' as an option into the square bracket to remove colours from this template for printing. 
                                    % Add 'espanol' as an option into the square bracket to change the date format of the Last Updated Text
\usepackage{polyglossia}
\usepackage{marvosym}
\usepackage{fontawesome5}

\begin{document}

\header{Patryk }{Choma}           % Your name
\lastupdated

%----------------------------------------------------------------------------------------
%	SIDEBAR SECTION  -- In the aside, each new line forces a line break
%----------------------------------------------------------------------------------------

\begin{aside}
%
\section{contatti}
Via Pasquale II 53
00168, Roma
Italia
~
{\Telefon} 346 392 4445 \\(\textbf{solo previa eMail})
~
{\Email} patryk.choma
@gmail.com
%
\section{web}
\faIcon{globe}\href{ch-pat.github.io/}{\underline{sito web}}
~
\faIcon{linkedin} \href{https://www.linkedin.com/in/p-choma/}{\underline{LinkedIn}}
~
\faIcon{github} \href{https://github.com/ch-pat}{\underline{GitHub}}
%
\section{lingue}
\textbf{Italiano} madrelingua
\textbf{Inglese} alto
%
\section{programmazione}
\textbf{Python}
Java
Go
%
\end{aside}

%----------------------------------------------------------------------------------------
%	SKILLS SECTION
%----------------------------------------------------------------------------------------

\section{Skills}
  \vspace{-0.4cm}
  \textbf{Git}, bash/terminal, SQL, HTML, CSS, Python, Java, Go, Javascript, Node.js, Selenium, Excel, Android, Linux, Windows, Google Cloud Platform.

%----------------------------------------------------------------------------------------
%	WORK EXPERIENCE SECTION
%----------------------------------------------------------------------------------------

\section{Esperienza}

\begin{entrylist}
    %------------------------------------------------
    \entry
    {2022--Now}
    {NTT Data}
    {Rome, Italy}
    {\jobtitle{Software Developer Engineer}\\
    Lavoro su Google cloud functions che raccolgono dati da un db spanner, scritto con framework express API e unit tested in nodejs; integration tests con python Behave per BDD.
    }
    %------------------------------------------------
    \entry
    {2021--2022}
    {Ericsson}
    {Rome, Italy}
    {\jobtitle{VMWare Operation Engineer}\\
    Gestione configurazioni di rete e troubleshooting stack cloud di VMWare.}
    %------------------------------------------------
\entry
  {2020--2021}
  {Freelancing}
  {Roma, Italia}
  {\jobtitle{Freelance Developer}\\
  Lavorato su due applicazioni desktop in python con semplici UI:
  \begin{itemize}
  \item Interfaccia per NAS Synology per gestione di file, creazione utenti, notifica e conferma di ricezione file.
  \item CRUD user-friendly per la modifica di file Excel usando Pandas.
  \end{itemize}
}
%------------------------------------------------
\entry
  {2018--2021}
  {Amministrazioni Martellucci}
  {Roma, Italia}
  {\jobtitle{Contabilità e supporto IT}\\
  Gestito, tracciato e documentato la contabilità. Migliorato il workflow con fogli Excel o script Python. Vedi il mio script in Selenium sul mio github repo "downbpiol".
}
%------------------------------------------------
\entry
  {2010--2020}
  {Lezioni private}
  {Roma, Italia}
  {\jobtitle{Tutor privato}\\
    Do lezioni private di matematica e inglese. Questo mi ha permesso di pagare i miei studi e mi ha insegnato la pazienza e la capacità di insegnare ad altri.
  }
\end{entrylist}

%----------------------------------------------------------------------------------------
%	EDUCATION SECTION
%----------------------------------------------------------------------------------------

\section{Formazione}

\begin{entrylist}
%------------------------------------------------
\entry
{2020--Oggi}
{Laurea Magistrale in Computer Science}
{Sapienza Università di Roma | Roma, Italia}
{Continuazione dei miei studi in CS. I corsi si concentrano principalmente su:
\begin{itemize}
    \item Data Science, Machine Learning e AI
    \item Computer Vision e Computer Graphics
    \item Cloud Computing e Big Data
\end{itemize}
}
\entry
{2016--2020}
{Laurea Triennale in Informatica}
{Sapienza Università di Roma | Roma, Italia}
{Ho appreso tutte le fondamenta dell'informatica, tra cui:
\begin{itemize}
    \item Algoritmi e Strutture Dati
    \item Progettazione di database relazionali
    \item Object Oriented Programming
\end{itemize}
La mia tesi di laurea è basata sul lavoro svolto sulla API del progetto "SeismoCloud", che interroga un database MySQL, scritta in Go e rilasciata attraverso Docker.
}
%------------------------------------------------
\end{entrylist}

%----------------------------------------------------------------------------------------
%	SOFT SKILLS SECTION
%----------------------------------------------------------------------------------------

\section{Abilità}
  \vspace{-0.4cm}
Ascolto, 
scrittura sintetica,
adattabilità,
pazienza,
desiderio di imparare,
pensiero logico,
problem solving,
affidabilità.
%----------------------------------------------------------------------------------------
%		INTERESTS SECTION
%----------------------------------------------------------------------------------------
\vspace{-0.4cm}
\section{Interessi}

\begin{entrylist}
%------------------------------------------------
\entry
{Professionali}
{- {\normalfont 
Continuare l'apprendimento, acquisire nuove abilità e lavorare su progetti utili e interessanti. Voglio continuare a concentrarmi sulle mie abilità di sviluppo back-end, ma anche avere l'opportunità di diventare uno sviluppatore più a tutto tondo}}
{}
{\vspace{-0.5cm}}
%------------------------------------------------
\end{entrylist}
\end{document}