% -- Encoding UTF-8 without BOM
% -- XeLaTeX => PDF (BIBER)

\documentclass[]{cv-style}          % Add 'print' as an option into the square bracket to remove colours from this template for printing. 
                                    % Add 'espanol' as an option into the square bracket to change the date format of the Last Updated Text
\usepackage{polyglossia}
\usepackage{marvosym}
\usepackage{fontawesome5}

\begin{document}

\header{Patryk }{Choma}           % Your name
\lastupdated

%----------------------------------------------------------------------------------------
%	SIDEBAR SECTION  -- In the aside, each new line forces a line break
%----------------------------------------------------------------------------------------

\begin{aside}
%
\section{contatti}
Via Pasquale II 53
00168, Roma
Italia
~
{\Telefon} 346 392 4445 \\(\textbf{solo previa eMail})
~
{\Email} patryk.choma
@gmail.com
%
\section{web}
\faIcon{globe} \href{https://ch-pat.github.io/}{\underline{sito web}}
~
\faIcon{linkedin} \href{https://www.linkedin.com/in/p-choma/}{\underline{LinkedIn}}
~
\faIcon{github} \href{https://github.com/ch-pat}{\underline{GitHub}}
%
\section{lingue}
\textbf{Italiano} madrelingua
\textbf{Inglese} alto
%
\section{programmazione}
\textbf{Python}
Node.js
Java
Go
%
\section{tecnologie}
\textbf{Google Cloud Platform}
Spanner, BigQuery, BigTable
Cloud Functions
Git, Bash, SQL
Jest, Express
Angular, HTML, CSS
Flask, Selenium, Behave
Android, Linux, Windows
%
\section{soft skills}
Ascolto
Scrittura sintetica
Adattabilità
Pazienza
Desiderio di imparare
Pensiero logico
Problem solving
Affidabilità
\end{aside}

%----------------------------------------------------------------------------------------
%	WORK EXPERIENCE SECTION
%----------------------------------------------------------------------------------------

\section{Esperienza}

\begin{entrylist}
%------------------------------------------------
\entry
{2022--Oggi}
{NTT Data}
{Rome, Italy}
{\jobtitle{Software Developer Engineer}\\
Lavoro nel progetto NextJenIoT di Assicurazioni Generali avendo a che fare con le black box installate nelle autovetture a fini assicurativi.\\
Il progetto vive nell'ecosistema di Google Cloud Platform: Le API sono deployate come Google Cloud Functions a i dati sono conservati su tabelle Spanner e BigQuery.\\
Ho sviluppato, progettato, deployato e mantenuto API (scritte in Python e Nodejs) che leggono ed interagiscono con tutti i tipi di dati di questi dispositivi, come dati gps e accelerazioni relativi a viaggi e incidenti.\\
Ho sviluppato test unitari e di integrazione che coprono tutti gli use case e ho documentato rigorosamente il mio lavoro.
}
%------------------------------------------------
\entry
{2021--2022}
{Ericsson}
{Rome, Italy}
{\jobtitle{VMWare Operation Engineer}\\
Gestione configurazioni di rete e troubleshooting stack cloud di VMWare.}
%------------------------------------------------
\entry
  {2020--2021}
  {Freelancing}
  {Roma, Italia}
  {\jobtitle{Freelance Developer}\\
  Lavorato su due applicazioni desktop in python con semplici UI:
  \begin{itemize}
  \item Interfaccia per NAS Synology per gestione di file, creazione utenti, notifica e conferma di ricezione file.
  \item CRUD user-friendly per la modifica di file Excel usando Pandas.
  \end{itemize}
}
%------------------------------------------------
\entry
  {2018--2021}
  {Amministrazioni Martellucci}
  {Roma, Italia}
  {\jobtitle{Contabilità e supporto IT}\\
  Gestito, tracciato e documentato la contabilità. Migliorato il workflow con fogli Excel o script Python. Vedi il mio script in Selenium sul mio github repo "downbpiol".
}
%------------------------------------------------
\entry
  {2010--2020}
  {Lezioni private}
  {Roma, Italia}
  {\jobtitle{Tutor privato}\\
    Do lezioni private di matematica e inglese. Questo mi ha permesso di pagare i miei studi e mi ha insegnato la pazienza e la capacità di insegnare ad altri.
  }
\end{entrylist}

%----------------------------------------------------------------------------------------
%	EDUCATION SECTION
%----------------------------------------------------------------------------------------

\section{Formazione}

\begin{entrylist}
%------------------------------------------------
\entry
{2020--Oggi}
{Laurea Magistrale in Computer Science}
{Sapienza Università di Roma | Roma, Italia}
{Continuazione dei miei studi in CS. I corsi si concentrano principalmente su:
\begin{itemize}
    \item Data Science, Machine Learning e AI
    \item Computer Vision e Computer Graphics
    \item Cloud Computing e Big Data
\end{itemize}
}
\entry
{2016--2020}
{Laurea Triennale in Informatica}
{Sapienza Università di Roma | Roma, Italia}
{Ho appreso tutte le fondamenta dell'informatica, tra cui:
\begin{itemize}
    \item Algoritmi e Strutture Dati
    \item Progettazione di database relazionali
    \item Object Oriented Programming
\end{itemize}
La mia tesi di laurea è basata sul lavoro svolto sulla API del progetto "SeismoCloud", che interroga un database MySQL, scritta in Go e rilasciata attraverso Docker.
}
%------------------------------------------------
\end{entrylist}

%----------------------------------------------------------------------------------------
%	SOFT SKILLS SECTION
%----------------------------------------------------------------------------------------

%----------------------------------------------------------------------------------------
%		INTERESTS SECTION
%----------------------------------------------------------------------------------------
\section{Interessi}

\begin{entrylist}
%------------------------------------------------
\entry
{} %{Professionali}
{{\normalfont 
Estendere il mio skillset, lavorare su progetti utili e interessanti con diverse tecnologie. Il mio interesse principale ricade sullo sviluppo di back-end, test e automazione, ma non mi dispiace provare qualcosa di diverso ogni tanto.
}}
{}
{\vspace{-0.5cm}}
%------------------------------------------------
\end{entrylist}
\end{document}